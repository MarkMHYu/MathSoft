\documentclass{ctexart}
\usepackage{amsmath, amsthm}
\pagestyle{plain}
\title{作业二: 函数}
\author{余明壕\\统计学 3190103127}
\begin{document}
\maketitle
选取的 Try it out 在书中的第48页,涉及函数、循环以及条件判断,实现功能为展示输入姓名,并询问所填名字是否正确并给出相应的回答。的现对代码进行展示:
\begin{verbatim}
#!/bin/bash

# function
yes_or_no(){
    echo "Is your name $* ?"
    while true
    do
        echo -n "Enter yes or not: "
        read x
        case "$x" in
            [yY] | [Yy][Ee][Ss] ) return 0;;
            [nN] | [nN][oO] )     return 1;;
            * )                   echo "Please answer yes or no."
        esac
    done
}

# main
echo "Original parameters are $*."

if yes_or_no "$1"
then
    echo "Hi $1, nice name."
else
    echo "Never mind."
fi
exit 0
\end{verbatim}
\section{代码分析}
代码分为两部分,函数与主体。
\subsection{函数}
函数部分主要通过 \verb|while| 循环与 \verb|case| 条件判断来实现。通过 \verb|echo| 与用户进行交互,设置 \verb|while| 的值恒为 \verb|true| 再通过 \verb|case| 的条件判断,使得用户不输入 
 \verb|yes| 或者 \verb|no| 将一直处于循环之中。
\subsection{主体}
主体部分则是调用函数 \verb|yes_or_no| 获取返回值,再通过 \verb|if| 条件判断进行相应的输出。
\section{运行结果}
首先通过 \verb|cd| 命令设置当前路径为shell脚本文件所在位置,之后输入
\begin{verbatim}
./my_name parameters
\end{verbatim}
运行程序。此处将 \verb|parameters| 取值为 \verb|Mark Yu|。测试结果如下:
\begin{verbatim}
# test1
Original parameters are Mark Yu.
Is your name Mark ?
Enter yes or not: ne
Please answer yes or no.
Enter yes or not: yo          
Please answer yes or no.
Enter yes or not: N
Never mind.
# test2
Original parameters are Mark Yu.
Is your name Mark ?
Enter yes or not: YeS
Hi Mark, nice name.
\end{verbatim}
\section{体会与感想}
代码理解和实现上来说并不困难,但是完成过程中仍给我造成了些麻烦,比如将 \verb|true| 打成了 \verb|True| 导致报错,并且在代码修改后没有重新编译导致错误一直存在,程序无法运行。不过我也因此学到了,
需要重视大小写以及编译。除此之外,通过代码的编写,我了解到了 \verb|case| 的多条件设置,读取参数的不同方式以及利用好循环体实现多次读取的操作。
\end{document}
